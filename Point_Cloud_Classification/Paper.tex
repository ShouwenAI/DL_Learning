应用于点云特征提取与分类的残差RNN网络

Abstract:
我们提出一种基于LSTM+Residual RNN网络的自动分类系统,实现从LIDAR点云数据(x,y,z,r,g,b)输入到分类结果的自动化处理。

Related Work:

Method:

1.第一部分网络
特征提取网络:
网络提取传统特征+网络自动抽象特征
特征提取网络对每个输入点实现从六维(x,y,z,r,g,b)数据输入到N维特征向量的映射。Residual RNN网络经过监督训练可以提取诸如法向量,spin图等具有语义的传统点云特征,同时也得到部分网络自己学习得到的抽象特征,将两部分特征并列结合得到最终的特征向量作为第二个网络的输入。

2.第二部分网络

网络特征的进一步抽象、融合,并进行分类。


semantic features:传统的,模板化、公式化的特征提取可以有效得到结构信息,但从六维向量到特征的映射过程中有不可避免的信息损失,且针对不同类型的点云数据(识别对象)无法做到因地制宜的自动调整。
RNN自动提取的features:使用RNN网络进行特征提取可以随场景自动调整,且信息损失少,但目前点云数据训练速度远低于图像数据,一个simple RNN对large-scale点云场景的训练收敛极慢。训练场景的数量不足可能导致网络过拟合,影响泛化能力。

我们的方法结合了二者优点:RNN features对信息提取的灵活性,全面性(信息损失少);semantic features给场景清晰定量的数学描述,有普适性,加快网络收敛。最后,第二部分网络Residual RNN可以再对既有特征,包括高阶和低阶特征进行融合、重组,并得到分类结果。


Experiment:

1.Stanford indoor dataset (12 classes)

2.German 某数据 (8 classes)

3.Tianjin 遥感数据 (4 classes)

与SVM,传统特征提取+CRF,普通RNN, etc 方法分别进行对比。

Conclusion:
